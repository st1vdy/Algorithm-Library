\documentclass{article}
\usepackage{fancyhdr}  % 页眉页脚
\usepackage{ctex}
\usepackage[margin=1in]{geometry}
\usepackage[Glenn]{fncychap}
\usepackage{listings}
\usepackage{color}
\usepackage{verbatim}	
\usepackage{amsmath}

\geometry{a4paper,left=2cm,right=2cm,top=2.5cm,bottom=2cm}

% 定义页眉页脚
\pagestyle{fancy}
\fancyhf{}
\fancyhead[C]{Algorithm Library by st1vdy}
\lfoot{}
\cfoot{\thepage}
\rfoot{}

\title{Algorithm Library}
\author{st1vdy}

\definecolor{dkgreen}{rgb}{0,0,0}
\definecolor{gray}{rgb}{0.5,0.5,0.5}
\definecolor{mauve}{rgb}{0.58,0,0.82}

\lstset{
    language=c++,                               % 设置默认语言
    breaklines,                                 % 自动将长的代码行换行排版
    extendedchars=false,                        % 解决代码跨页时,章节标题,页眉等汉字不显示的问题
    backgroundcolor=\color{white},              % 背景颜色
    keywordstyle=\color{blue}\bfseries,         % 关键字颜色
    identifierstyle=\color[rgb]{0,0,0},         % 普通标识符颜色
    commentstyle=\color[rgb]{0,0.6,0},          % 注释颜色
    stringstyle=\color[rgb]{0.58,0,0.82},       % 字符串颜色
    showstringspaces=false,                     % 不显示字符串内的空格
    numbers=left,                               % 显示行号
    captionpos=t,                               % title在上方(在bottom即为b)
    frame=single,                               % 设置代码框形式
    rulecolor=\color[rgb]{0,0,0},               % 设置代码框颜色
}

\begin{document}
\begin{titlepage}
\maketitle
\thispagestyle{empty}
\pagebreak
\pagestyle{plain}
\tableofcontents
\end{titlepage}
%\twocolumn  % 是否需要分栏显示
\newpage % 另起一页
\section{多项式} %一级标题
\subsection{FFT - tourist}
\lstinputlisting{polynomial/fft_tourist.cpp}
\subsection{形式幂级数}

\section{数论}
\subsection{简单的防爆模板}
\lstinputlisting{number_theory/mod.cpp}
\subsection{筛法}
\subsubsection{线性素数筛}
\lstinputlisting{number_theory/sieve.cpp}
\subsubsection{线性欧拉函数筛}
\lstinputlisting{number_theory/phi_sieve.cpp}
\subsubsection{线性约数个数函数筛}
\lstinputlisting{number_theory/div_sieve.cpp}
\subsubsection{线性素因子个数函数筛}
\lstinputlisting{number_theory/pdiv_sieve.cpp}
\subsubsection{线性约数和函数筛}
\lstinputlisting{number_theory/sigma_sieve.cpp}
\subsubsection{线性莫比乌斯函数筛}
\lstinputlisting{number_theory/mobius_sieve.cpp}
\subsection{扩展欧几里得}
\subsubsection{线性同余方程最小非负整数解}
exgcd求 $ax+by=c$ 的最小非负整数解详解:
\begin{enumerate}
    \item 求出 $a,b$ 的最大公约数 $g=\gcd(a,b)$ ,根据裴蜀定理检查是否满足 $c\% g=0$ ,不满足则无解;
    \item 调整系数 $a,b,c$ 为 $a'=\frac{a}{g},b'=\frac{b}{g},c'=\frac{c}{g}$ ,这是因为 $ax+by=c$ 和 $a'x+b'y=c'$ 是完全等价的;
    \item 实际上exgcd求解的方程是 $a'x+b'y=1$ ,求解前需要注意让系数 $a',b'\geq 0$ (举个例子,如果系数 $b'$ 原本 $<0$ ,我们可以翻转 $b'$ 的符号然后令解 $(x,y)$ 为 $(x,-y)$ ,但是求解的时候要把 $y$ 翻回来);
    \item 我们通过exgcd求出一组解 $(x_0,y_0)$ ,这组解满足 $a'x_0+b'y_0=1$ ,为了使解合法我们需要令 $x_0=c'x_0,y_0=c'y_0$ ,于是有 $a'(c'x_0)+b'(c'y_0)=c''$ ;
    \item 考虑到 $a'x_0+b'y_0=1$ 等价于同余方程 $a'x_0\equiv 1\pmod{b'}$ ,因此为了求出最小非负整数解,我们最后还需要对 $b'$ 取模;
    \item 最后注意特判 $c'=0$ 的情况,如果要求解 $y$ 且系数 $b$ 发生了翻转,将其翻转回来。
\end{enumerate}

\lstinputlisting{number_theory/exgcd.cpp}
\subsection{欧拉定理}

$$
a^b\equiv
\begin{cases}
a^{b\bmod\varphi(p)},\,&\gcd(a,\,p)=1\\
a^b,&\gcd(a,\,p)\ne1,\,b<\varphi(p)\\
a^{b\bmod\varphi(p)+\varphi(p)},&\gcd(a,\,p)\ne1,\,b\ge\varphi(p)
\end{cases}
\pmod p
$$

\subsection{欧拉函数}
\subsection{中国剩余定理}
\subsubsection{CRT}
\lstinputlisting{number_theory/crt.cpp}
\subsubsection{EXCRT}
\lstinputlisting{number_theory/excrt.cpp}
\subsection{BSGS}
\subsection{迪利克雷卷积}

$$
\begin{aligned}
g(1)S(n)=\sum_{i=1}^n(f*g)(i)-\sum_{i=2}^ng(i)S(\lfloor\frac{n}{i}\rfloor)
\end{aligned}
$$

\subsection{杜教筛}

$$
\begin{aligned}
(f*g)(n)=\underset{d|n}{\sum} f(d)g(\frac{n}{d})=\underset{xy=n}{\sum} f(x)g(y)
\end{aligned}
$$

\section{线性代数}
\subsection{高斯-约旦消元法}
\lstinputlisting{linear_algebra/gauss.cpp}
\subsection{高斯消元法-bitset}
\lstinputlisting{linear_algebra/gauss_bitset.cpp}
\subsection{线性基}
\lstinputlisting{linear_algebra/linear_basis.cpp}
\subsection{矩阵树定理}
\lstinputlisting{linear_algebra/matrix_tree.cpp}
\subsection{LGV引理}

\section{组合数学}
\subsection{组合数预处理}
\lstinputlisting{combinatorics/binom.cpp}
\subsection{卢卡斯定理}
\subsection{小球盒子模型}
\subsection{斯特林数}
\subsubsection{第一类斯特林数}

第一类斯特林数 $\begin{bmatrix}n\\k\end{bmatrix}$ 表示将 $n$ 个不同元素划分入 $k$ 个非空圆排列的方案数。

$$
\begin{bmatrix}
n \\ k
\end{bmatrix}
=
\begin{bmatrix}
n-1 \\ k-1
\end{bmatrix}
+
(n-1)\begin{bmatrix}
n-1 \\ k
\end{bmatrix}
$$

边界是 $\begin{bmatrix}0\\0\end{bmatrix}=1$ 。

第一类斯特林数三角形,从 s(1, 1) 开始:

$$
\begin{matrix}
1 \\
1 & 1 \\
2 & 3 & 1 \\
6 & 11 & 6 & 1 \\
24 & 50 & 35 & 10 & 1 \\
120 & 274 & 225 & 85 & 15 & 1 \\
720 & 1764 & 1624 & 735 & 175 & 21 & 1 \\
5040 & 13068 & 13132 & 6769 & 1960 & 322 & 28 & 1 \\
40320 & 109584 & 118124 & 67284 & 22449 & 4536 & 546 & 36 & 1 \\
362880 & 1026576 & 1172700 & 723680 & 269325 & 63273 & 9450 & 870 & 45 & 1 \\
\end{matrix}
$$


\subsubsection{第二类斯特林数}

第二类斯特林数 $\begin{Bmatrix} n\\k \end{Bmatrix}$ 表示将 $n$ 个不同元素划分为 $k$ 个非空子集的方案数。

$$
\begin{aligned}
\begin{Bmatrix} n\\k \end{Bmatrix}
=
\begin{Bmatrix} n-1\\k-1 \end{Bmatrix}
+
k\begin{Bmatrix} n-1\\k \end{Bmatrix}
\end{aligned}
$$

边界是 $\begin{Bmatrix}0\\0\end{Bmatrix}=1$ 。

基于容斥原理的递推方法:

$$
\begin{aligned}
\begin{Bmatrix} n\\k \end{Bmatrix} =
\frac{1}{k!}\sum^k_{i=0}(-1)^i\binom{k}{i}(k-i)^n
\end{aligned}
$$

第二类斯特林数三角形,从 S(1, 1) 开始:

$$
\begin{matrix}
1 \\
1 & 1 \\
1 & 3 & 1 \\
1 & 7 & 6 & 1 \\
1 & 15 & 25 & 10 & 1 \\
1 & 31 & 90 & 65 & 15 & 1 \\
1 & 63 & 301 & 350 & 140 & 21 & 1 \\
1 & 127 & 966 & 1701 & 1050 & 266 & 28 & 1 \\
1 & 255 & 3025 & 7770 & 6951 & 2646 & 462 & 36 & 1 \\
1 & 511 & 9330 & 34105 & 42525 & 22827 & 5880 & 750 & 45 & 1 \\
\end{matrix}
$$

\section{博弈论}


\section{图论}
\subsection{并查集}
\lstinputlisting{graph/dsu.cpp}
\subsection{最小树形图}
\lstinputlisting{graph/zhuliu.cpp}
\subsection{最近公共祖先}
\lstinputlisting{graph/lca.cpp}
\subsection{强连通分量}
\lstinputlisting{graph/scc.cpp}
\subsection{最大流}
\lstinputlisting{graph/maxflow_acl.cpp}
\subsection{最小费用最大流}
\lstinputlisting{graph/mincostmaxflow_acl.cpp}
\subsection{全局最小割}

\subsection{二分图最大权匹配}
\lstinputlisting{graph/km.cpp}
\subsection{一般图最大匹配}
\subsection{2-sat}
\subsection{最大团}
\lstinputlisting{graph/max_clique.cpp}

\section{数据结构}
\subsection{树状数组}

\section{字符串}
\subsection{KMP}
\lstinputlisting{string/kmp.cpp}
\subsection{Z-Function}
\subsection{Manacher}
\lstinputlisting{string/manacher.cpp}
\subsection{Trie}
\lstinputlisting{string/trie.cpp}
\subsection{01-Trie}
\lstinputlisting{string/xortrie.cpp}

\section{计算几何}

\section{杂项}
\subsection{蔡勒公式}

\end{document}